%%% Для сборки выполнить 2 раза команду: pdflatex <имя файла>

\documentclass[a4paper,12pt]{article}

\usepackage{ucs}
\usepackage[utf8x]{inputenc}
\usepackage[russian]{babel}
%\usepackage{cmlgc}
\usepackage{graphicx}
\usepackage{listings}
\usepackage{xcolor}
%\usepackage{courier}

\makeatletter
\renewcommand\@biblabel[1]{#1.}
\makeatother

\newcommand{\myrule}[1]{\rule{#1}{0.4pt}}
\newcommand{\sign}[2][~]{{\small\myrule{#2}\\[-0.7em]\makebox[#2]{\it #1}}}

% Поля
\usepackage[top=20mm, left=30mm, right=10mm, bottom=20mm, nohead]{geometry}
\usepackage{indentfirst}

% Межстрочный интервал
\renewcommand{\baselinestretch}{1.50}


\begin{document}

%%%%%%%%%%%%%%%%%%%%%%%%%%%%%%%
%%%                         %%%
%%% Начало титульного листа %%%

\thispagestyle{empty}
\begin{center}


\renewcommand{\baselinestretch}{1}
{\large
{\sc Петрозаводский государственный университет\\
Институт математики и информационных технологий\\
	Кафедра информатики и математического обеспечения
}
}

\end{center}


\begin{center}
%%%%%%%%%%%%%%%%%%%%%%%%%
%
% Раскомментируйте (уберите знак процента в начале строки)
% для одной из строк типа направления  - бакалавриат/
% магистратура и для одной из
% строк Вашего направление подготовки
%

% 09.03.02 - Информационные системы и технологии \\
% 09.03.04 - Программная инженерия \\
% Направление подготовки магистратуры \\
% 01.04.02 - Прикладная математика и информатика \\
% 09.04.02 - Информационные системы и  технологии \\
%
% 
  %%%%%%%%%%%%%%%%%%%%%%
Направление подготовки бакалавриата \\
01.03.02 Прикладная математика и информатика 
\end{center}

\vfill

\begin{center}

%%% Название работы %%%
	{\Large \sc Помощник-тренажер по дисциплинам первого курса ПМиИ} \\
	(промежуточный)
\end{center}

\medskip

\begin{flushright}
\parbox{11cm}{%
\renewcommand{\baselinestretch}{1.2}
\normalsize
	Выполнил:\\
%%% ФИО студента %%%
студент 2 курса группы 22204
\begin{flushright}
А. Р. Артамонов \sign[подпись]{4cm}
\end{flushright}
%%%%%%%%%%%%%%%%%%%%%%%%%
% девушкам применять "Выполнила" и "студентка"
%%%%%%%%%%%%%%%%%%%%%%%%%

% Если руководителей два - то раскомментровать строку про второго руководителя и применть "Руководители:"

Руководители:\\
%%% степень, звание ФИО научного руководителя %%%
% Первый руководитель 
Ю. А. Богоявленский, к.т.н., доцент \\
\begin{flushright}
\sign[подпись]{4cm}
\end{flushright}


Д.Б. Чистяков \\
\begin{flushright}
\sign[подпись]{4cm}
\end{flushright}



Итоговая оценка
\begin{flushright}
  \sign[оценка]{4cm}
\end{flushright}
}
\end{flushright}

\vfill

\begin{center}
\large
    Петрозаводск --- 2024
\end{center}

%%% Конец титульного листа  %%%
%%%                         %%%
%%%%%%%%%%%%%%%%%%%%%%%%%%%%%%%

%%%%%%%%%%%%%%%%%%%%%%%%%%%%%%%%
%%%                          %%%
%%% Содержание               %%%

\newpage

\tableofcontents

%%% Содержание              %%%
%%%                         %%%
%%%%%%%%%%%%%%%%%%%%%%%%%%%%%%%


%%%%%%%%%%%%%%%%%%%%%%%%%%%%%%%%
%%%                          %%%
%%% Введение                 %%%

%%% В введении Вы должны описать предметную область, с которой связана %%%
%%% Ваша работа, показать её актуальность, вкратце определить цель     %%%
%%% исследования/разработки					       %%%


\newpage
\section{Введение}
\addcontentsline{toc}{section}{Введение}

 Каждый год новоприбывшие первокурсники сталкиваются с проблемой изучения нового, ранее не затронутого материала. Эта проблема влечет за собой множество других проблем, таких как чрезмерная нагрузка, неуспеваемость. Студентам необходимо сохранять высокий темп изучения материала, при этом не всегда есть возможность изучить какую-либо тему вместе с преподавателем. Именно по этой причине разрабатываемое нами приложение-тренажер имеет высокую актуальность.\\

 Почти каждый студент владеет мобильным устройством с возможностью постоянного или частичного подключения к сети. Этот фактор сыграл большую роль в выборе разрабатываемого программного обеспечения. \\
 
 Цель нашей работы - решить проблемы неуспеваемости и плохого понимания студентами первого курса ПМиИ нового материала по соответствующим дисциплинам этого курса.\\

 Объектом исследования данной курсовой работы является определение мобильных приложений, как способа изучения нового материала и его запрепления.\\
 
 Предметом исследования курсовой работы станет практическое примение приложения-тренажера в качестве помощника студентам первого курса ПМиИ.\\
 
 Для достижения нашей цели мы выделили несколько задач:\\
 1) Изучить новый язык программировния для разработки мобильных приложений.\\
 2) Изучить дисциплины и темы первого курса ПМиИ.\\
 3) Реализовать полученные знания в приложении-тренажере.\\

 Данная курсовая работа состоит из ...\\

 Мы предполагаем, что наше приложение-тренажер поможет студентам первого курса в изучении новых дисциплин и более быстрого освоения своей специальности. Наше приложение позволит студентам опробовать свои знания на практике и поможет убрать пробелы в изучении тех или иных тем.\\ 

%%%                          %%%
%%%%%%%%%%%%%%%%%%%%%%%%%%%%%%%%

\newpage
\chapter{Глава 1}
\addcontentsline{toc}{chapter}{Введение}
\end{document}
